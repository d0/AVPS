\documentclass{scrartcl}
\usepackage[ngerman]{babel}
\usepackage[utf8x]{inputenc}
\usepackage{listings}

\lstset{language=C, 
		basicstyle=\footnotesize, 
		showstringspaces=false
		numbers=left,
		numberstyle=\footnotesize,
		breaklines=true}

\usepackage{booktabs}
\usepackage{hyperref}

\begin{document}
\title{Übung 2}
\author{Evenij Belikov\\Jan Birkholz\\Dominik Oepen}
\maketitle

\section*{Aufgabe 2.1}

\lstinputlisting{uebung2.c}

Die Messwerte der einzelnen Parametersätze sind in \autoref{messwerte} aufgeführt. Es wurde mit jeder Methode jeweils mit 2, 4 und 8 Prozessen und einer Arraygröße von 1024, 1048576 und 33554432 Elementen gemessen. Die lokale Ausführung fand auf einem PC mit einem Intel Core 2 Duo E8400 (3 GHz) und 4 GB DDR-2 Ram statt. Bei der verteilten Messung wurde zusätzlich ein Laptop mit Intel Core 2 Duo P9600 (2,53 GHz) und ebenfalls 4 GB Ram verwendet. Beide Knoten waren dabei via 100 MBit Ethernet verbunden.


\section*{Aufgabe 2.2}

Verwendet wurde OpenMPI aus den Repositories von Ubuntu 10.10. In
dieser MPI\-Distribution ist \texttt{MPI\_SEND} als gepuffertes Senden
implementiert. Dies lässt sich über den in Aufgabe 2.3 nachgestellten
Deadlock verifizieren. Wird dort die Methode \texttt{MPI\_SEND} verwendet so
kommt es zu keinem Deadlock. Dies kann nur bei der der Verwendung
eines gepufferten Sendekommandos zutreffen, da in diesem Fall das
der sendende Prozess nicht blockiert, sondern das \texttt{MPI\_RECV} aufrufen
kann.

Die Gegenprobe erfolgt über die Verwendung \texttt{MPI\_SSEND} in Aufgabe
2.3, was in der Tat zu einem Deadlock führt.

\section*{Aufgabe 2.3}

%Beide Situation werden in \lstinline$uebung2_3.c$ nachgestellt. Der Quellcode wird mit:
%
%\begin{lstlisting}
%	mpicc -O3 deadlock.c -o deadlock -Wall -Wextra
%\end{lstlisting} 
%
%übersetzt. Die Aufrufstruktur des Programms ist die Folgende:
%\begin{lstlisting}
%	mpirun -np N ./deadlock --problem=prob --solution=sol
%\end{lstlisting}
%
%Der Parameter \texttt{problem} bestimmt dabei welche der beiden Situation nachgestellt werden soll. Zulässige Werte sind \texttt{deadlock} und \texttt{order}.

Der Parameter \texttt{solution} legt fest ob und welche
Lösungsstrategie verwendet werden soll. Erlaubt sind hier die Zahlen
1, 2 und 3.

\lstinputlisting{uebung2_3.c}

Im Deadlock-Szenario führt Lösung 1 zur Verwendung eines gepufferten
Sendemechanismus (\texttt{MPI\_BSEND}). Lösung 2 führt zur Vertauschung der
Reihenfolge von \texttt{MPI\_SEND} und \texttt{MPI\_RECV} beim zweiten Prozess.

Im Szenario der vertauschten Nachrichtenreihenfolge führen Lösung 1
und 2 zur Synchronisation mittels \texttt{MPI\_BARRIER} and unterschiedlichen Stellen. In beiden Fällen wird sichergestellt, dass Prozess 3 die Nachricht von Prozess 1 erhalten hat bevor er das zweite \texttt{MPI\_RECV} aufruft. Lösung 3 besteht in der synchronen Versendung der Nachrichten mittels \texttt{MPI\_SSEND}. Dadurch wird die Nachrichtenreihenfolge beibehalten.
\section*{Aufgabe 2.4}

\begin{appendix}
\section{Messwerte}\label{messwerte}
\begin{table}[htp]
\begin{tabular}{lllcc}
\toprule
np	& method& size		& Durchschnittliche Dauer & Standardabweichung \\
\midrule
2	& send	& 1024		& 6.37e-05	& 7.50067330311e-05 \\
2	& send	& 1048576	& 0.0084882	& 0.00189346570077 \\
2	& send	& 33554432	& 0.2137896	& 0.0167108683509 \\
2	& ssend	& 1024		& 0.0001677	& 0.000262018720705 \\
2	& ssend	& 1048576	& 0.0071808	& 0.00164163027506 \\
2	& ssend	& 33554432	& 0.2403283	& 0.02819272542 \\
2	& bsend	& 1024		& 0.0011329	& 0.00213817297944 \\
2	& bsend	& 1048576	& 0.0154501	& 0.00336053446493 \\
2	& bsend	& 33554432	& 0.3842997	& 0.0357175572626 \\
4	& send	& 1024		& 0.0010367	& 0.0003890066966 \\
4	& send	& 1048576	& 0.0137986	& 0.00169043835735 \\
4	& send	& 33554432	& 0.3312539	& 0.02545613736 \\
4	& ssend	& 1024		& 0.0016339	& 0.000230677458803 \\
4	& ssend	& 1048576	& 0.0147138	& 0.00224142975799 \\
4	& ssend	& 33554432	& 0.3478213	& 0.0143619781301 \\
4	& bsend	& 1024		& 0.0010497	& 8.47007083796e-05 \\
4	& bsend	& 1048576	& 0.0221432	& 0.00250709983048 \\
4	& bsend	& 33554432	& 0.5389012	& 0.086010165642 \\
8	& send	& 1024		& 0.0026242	& 0.000991653245848 \\
8	& send	& 1048576	& 0.023807	& 0.00355234826558 \\
8	& send	& 33554432	& 0.5304979	& 0.0383064658966 \\
8	& ssend	& 1024		& 0.0049845	& 0.00204005736439 \\
8	& ssend	& 1048576	& 0.0289121	& 0.00360859206478 \\
8	& ssend	& 33554432	& 0.6618364	& 0.0552136746182 \\
8	& bsend	& 1024		& 0.003448	& 0.00110447489786 \\
8	& bsend	& 1048576	& 0.0405434	& 0.0078483432927 \\
8	& bsend	& 33554432	& 0.965756	& 0.0803113744783 \\
\bottomrule
\end{tabular}
\label{tab:lokal}
\caption{Messergebniss bei lokaler Ausführung}
\end{table}

\begin{table}[htp]
\begin{tabular}{lllcc}
\toprule
np	& method& size		& Durchschnittliche Dauer & Standardabweichung \\
\midrule
2	& send	& 1024		& 0.0050274	& 0.0148725333767 \\
2	& send	& 1048576	& 0.0176759	& 0.0279778427133 \\
2	& send	& 33554432	& 0.2135077	& 0.0148670983588 \\
2	& ssend	& 1024		& 6.85e-05	& 1.20415945788e-06 \\
2	& ssend	& 1048576	& 0.0077274	& 0.00201575138348 \\
2	& ssend	& 33554432	& 0.2145943	& 0.0157753960714 \\
2	& bsend	& 1024		& 7.72e-05	& 1.09617516848e-05 \\
2	& bsend	& 1048576	& 0.012425	& 0.00159733164997 \\
2	& bsend	& 33554432	& 0.3326128	& 0.00620659395804 \\
4	& send	& 1024		& 0.0018878	& 0.000510647197192 \\
4	& send	& 1048576	& 0.1258536	& 0.00138760521763 \\
4	& send	& 33554432	& 4.0216421	& 0.0386633531951 \\
4	& ssend	& 1024		& 0.0017445	& 0.000744792756409 \\
4	& ssend	& 1048576	& 0.127911	& 0.00242306574405 \\
4	& ssend	& 33554432	& 4.0221775	& 0.0260878917479 \\
4	& bsend	& 1024		& 0.0015531	& 0.000385118540192 \\
4	& bsend	& 1048576	& 0.1302235	& 0.00347815632915 \\
4	& bsend	& 33554432	& 4.1590669	& 0.078720831001 \\
8	& send	& 1024		& 0.003105	& 0.000588689561654 \\
8	& send	& 1048576	& 0.1146384	& 0.00143346567451 \\
8	& send	& 33554432	& 3.5846085	& 0.0276869637348 \\
8	& ssend	& 1024		& 0.0045598	& 0.00212628369697 \\
8	& ssend	& 1048576	& 0.1150149	& 0.00266901223114 \\
8	& ssend	& 33554432	& 3.6479422	& 0.0249198420212 \\
8	& bsend	& 1024		& 0.0030144	& 0.000674932026207 \\
8	& bsend	& 1048576	& 0.1227474	& 0.00620574171554 \\
8	& bsend	& 33554432	& 3.8256719	& 0.108386780645 \\
\bottomrule
\end{tabular}
\label{tab:verteilt}
\caption{Messergebniss bei verteilter Ausführung}
\end{table}
\end{appendix}
\end{document}