\documentclass{scrartcl}
\usepackage[ngerman]{babel}
\usepackage[utf8x]{inputenc}
\usepackage{listings}

\lstset{language=Erlang, 
		basicstyle=\footnotesize, 
		showstringspaces=false
		numbers=left,
		numberstyle=\footnotesize,
		breaklines=true}

\usepackage{booktabs}
\usepackage[colorlinks=false, pdfborder={0 0 0}]{hyperref}

%Pictures
%\usepackage{graphicx} %for pictures
%\usepackage{epstopdf} %for eps pictures in pdf
%\begin{figure}
%\includegraphics[scale=1]{send_recv_bwlf_3.eps}
%\end{figure}

\usepackage{amsthm, amssymb, amsfonts}
\newtheorem{lem}{Lemma}

\begin{document}
\title{Übung 6}
\author{Evenij Belikov\\Jan Birkholz\\Dominik Oepen}
\maketitle

\section*{Aufgabe 6.1}

\section*{Aufgabe 6.2}
Wir stellen ein Beispiel ähnlich dem aus der Vorlesung nach, allerdings mit drei Prozessen,
welche alle bidirektional miteinander verbunden sind.

\subsection*{Quellcode}
\lstinputlisting{chandy.erl}

\subsection*{Programmdurchlauf}
\begin{verbatim}
Erlang R13B03 (erts-5.7.4) [source] [64-bit] [smp:4:4] [rq:4] [async-threads:0] [hipe] [kernel-poll:false]

Eshell V5.7.4  (abort with ^G)
1> c(chandy).
./chandy.erl:39: Warning: variable 'Rest' is unused
{ok,chandy}
2> chandy:start().
Passive Process started: PID:<0.42.0>
Passive Process started: PID:<0.43.0>
Active process started. PID: <0.44.0>
<0.44.0>
<0.44.0> is connected with <0.42.0>
<0.44.0> is connected with <0.43.0>
   
Process self is starting the Chandy-Lamport algorithm
<0.42.0>: Received list of partners
<0.43.0>: Received list of partners
<0.42.0> is connected with <0.44.0>
<0.43.0> is connected with <0.44.0>
<0.42.0> is connected with <0.43.0>
<0.43.0> is connected with <0.42.0>
   
   
<0.42.0> received marker from <0.44.0>
<0.43.0> received marker from <0.44.0>
Process self is starting the Chandy-Lamport algorithm
Process self is starting the Chandy-Lamport algorithm
<0.43.0> has received a marker from process <0.42.0>
<0.44.0> has received a marker from process <0.42.0>
<0.42.0> has received a marker from process <0.43.0>
<0.43.0> has received markers from all other processes
<0.44.0> has received a marker from process <0.43.0>
<0.42.0> has received markers from all other processes
<0.43.0> has finished taking a snapshot
<0.44.0> has received markers from all other processes
<0.42.0> has finished taking a snapshot
{state,<0.43.0>,{1213,23}}
<0.44.0> has finished taking a snapshot
{state,<0.42.0>,{1815,50}}
{state,<0.44.0>,{0,900}}
{msg,<0.42.0>,{185,0}}
{msg,<0.43.0>,{124,0}}
3> 
BREAK: (a)bort (c)ontinue (p)roc info (i)nfo (l)oaded
       (v)ersion (k)ill (D)b-tables (d)istribution
a
\end{verbatim}

\section*{Aufgabe 6.3}
Zu zeigen: $ e \rightarrow e' \Longleftrightarrow  V(e) < V(e') $

\begin{lem}\label{lem:1}
$ e \rightarrow e' \Longrightarrow  V(e) < V(e') $
\end{lem}

\begin{proof}
Nach HB1 existiert ein Prozess $ p_{i}: e \rightarrow_{i} e' $, was wiederum nach der Definition der Totalen Ordnung heißt: e findet vor e' in $ p_{i} $ statt.
Wenn es so einen Prozess gibt dann muss $V(e) < V(e')$ denn nach VC2 wird vor Eintritt des Ereignisses der Zeitstempel erhöht.
\end{proof}

\begin{lem}\label{lem:2}
$V(e) < V(e') \Longrightarrow e \rightarrow e' $
\end{lem}

\begin{proof}
Die Vektoruhr von e ist kleiner als die von e'. Damit das passieren kann muss entweder nach VC2 ein Prozess $p_{i}$ vor einem Ereignis einen Zeitstempel erhöht haben (\ref{item:part1}), oder nach VC4 eine Nachricht erhalten haben die einen größeren Zeitstempel hat (\ref{item:part2}).
\begin{enumerate}
\item \label{item:part1} Wegen VC2 folgt auch das jeweils ein Ereignis stattgefunden hat (nach dem Erhöhen). Da das Erhöhen vor dem 2. Ereignis stattgefunden haben muss, gibt es entweder einen Prozess (HB1) $p_{i}$ in dem $e \rightarrow e'$ oder mehrere hintereinander (HB3).
\item \label{item:part2} Da nach HB2 das Empfangen nach dem Senden stattfindet, muss es ein Sendereignis e geben welches vor dem Empfangen von e' lag.
\end{enumerate}
\end{proof}

Aus Lemma 1 und Lemma 2 folgt: $ e \rightarrow e' \Longleftrightarrow  V(e) < V(e') $. \qed

%\qedhere
%Der Parameter \texttt{solution} legt fest ob und welche Lösungsstrategie verwendet werden soll. Erlaubt sind hier die Zahlen 1, 2 und 3.

%\begin{appendix}
%\section{Messwerte}\label{messwerte}
%\lstinputlisting{messungen.txt}
%\end{appendix}
\end{document}